\begin{tomluoc}
Đề tài thực tập này hướng đến việc nghiên cứu và triển khai 
\textbf{``Hệ thống Tự động hóa Truyền thông và Thương mại Điện tử tích hợp Chatbot Zalo''}. 
Mục tiêu chủ đạo là xây dựng một giải pháp toàn diện, hỗ trợ doanh nghiệp nâng cao hiệu quả trong 
hoạt động \textbf{Marketing}, bán hàng và chăm sóc khách hàng thông qua các công cụ tự động hóa.  

Hệ thống được định hình với ba thành phần trọng tâm:  

\begin{enumerate}
    \item \textbf{Tự động hóa truyền thông}: Thiết kế quy trình tự động tạo ra nội dung truyền thông 
    (video, hình ảnh, bài viết) và lập lịch đăng tải lên các kênh mạng xã hội một cách đồng bộ.
    
    \item \textbf{Website thương mại điện tử}: Phát triển một trang bán điện thoại di động dựa trên 
    nền tảng WordPress, tích hợp đầy đủ các chức năng cơ bản như quản lý sản phẩm, giỏ hàng, đơn hàng 
    và thanh toán.
    
    \item \textbf{Chatbot Zalo}: Xây dựng và cấu hình Chatbot có khả năng tư vấn, trả lời thắc mắc và 
    hỗ trợ khách hàng liên tục 24/7, đồng thời kết nối trực tiếp với hệ thống Website thương mại điện tử.
\end{enumerate}

Trong quá trình triển khai, các công cụ \textbf{N8N} và \textbf{Make.com} được ứng dụng để liên kết và 
tự động hóa toàn bộ quy trình, từ khâu tạo và đăng tải nội dung, quản trị Website đến việc tích hợp dữ liệu 
và giao tiếp qua Chatbot.  

Bên cạnh đó, đề tài còn tập trung đánh giá hiệu quả thực tế của hệ thống trong việc 
\textbf{tiết kiệm chi phí, rút ngắn thời gian vận hành, tăng tính chuyên nghiệp} và 
\textbf{cải thiện trải nghiệm khách hàng}, đặc biệt phù hợp với các \textbf{doanh nghiệp vừa và nhỏ} 
hoạt động trong lĩnh vực kinh doanh sản phẩm công nghệ tại thị trường Việt Nam.  

\end{tomluoc}