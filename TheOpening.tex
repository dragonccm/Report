\begin{abstract}
\section{LÍ DO CHỌN ĐỀ TÀI}
Trong bối cảnh chuyển đổi số đang diễn ra mạnh mẽ, các doanh nghiệp ngày càng
chú trọng đến việc tối ưu hoá hoạt động truyền thông và thương mại điện tử nhằm nâng
cao hiệu quả kinh doanh. Zalo hiện là một trong những nền tảng mạng xã hội phổ biến
tại Việt Nam, với lượng người dùng lớn và mức độ tương tác cao. Việc ứng dụng Chatbot
Zalo giúp doanh nghiệp tự động hóa quá trình chăm sóc khách hàng, quảng bá sản phẩm,
xử lý đơn hàng và hỗ trợ bán hàng 24/7, từ đó tiết kiệm chi phí, giảm tải nhân sự và nâng
cao trải nghiệm khách hàng.
Xuất phát từ nhu cầu thực tiễn đó, đề tài “Hệ thống tự động hóa truyền thông và
thương mại điện tử cho doanh nghiệp ứng dụng Chatbot Zalo” được lựa chọn với mong
muốn nghiên cứu, xây dựng giải pháp ứng dụng công nghệ vào hoạt động kinh doanh,
hỗ trợ doanh nghiệp mở rộng thị trường, tăng khả năng cạnh tranh và bắt kịp xu hướng
số hóa.
\section{MỤC TIÊU VÀ PHẠM VI THỰC TẬP}
\subsection{Mục tiêu} 
Thông qua đề tài “Hệ thống Tự động hóa Truyền thông và Thương mại Điện tử cho Doanh nghiệp ứng dụng Chatbot Zalo”, tôi mong muốn đạt được:  

\begin{itemize}
    \item \textbf{Tự động hóa quy trình truyền thông}: Giúp doanh nghiệp dễ dàng đăng tải, quản lý và phân phối nội dung trên nhiều kênh khác nhau, từ đó tiết kiệm thời gian và nhân lực.
    \item \textbf{Nâng cao hiệu quả chăm sóc khách hàng}: Ứng dụng Chatbot Zalo nhằm phản hồi khách hàng nhanh chóng, chính xác, hỗ trợ 24/7, cải thiện trải nghiệm và tăng mức độ hài lòng.
    \item \textbf{Hỗ trợ bán hàng trực tuyến}: Cung cấp kênh tư vấn, giới thiệu sản phẩm và xử lý đơn hàng trực tiếp trên nền tảng Zalo, góp phần thúc đẩy doanh số bán hàng.
    \item \textbf{Tối ưu chi phí và nguồn lực}: Giúp doanh nghiệp, đặc biệt là doanh nghiệp vừa và nhỏ, tiết kiệm chi phí vận hành và giảm tải công việc lặp lại cho nhân sự.
    \item \textbf{Nâng cao năng lực cạnh tranh}: Tạo lợi thế cho doanh nghiệp trong xu thế chuyển đổi số, bắt kịp các xu hướng công nghệ mới và mở rộng thị trường kinh doanh.
\end{itemize}

\subsection{Phạm vi thực hiện} 
\begin{itemize}
    \item \textbf{Địa điểm}: Thành phố Cần Thơ.  
    \item \textbf{Đối tượng nghiên cứu}: Hệ thống Chatbot Zalo và các công cụ hỗ trợ truyền thông, chăm sóc khách hàng, thương mại điện tử.  
    \item \textbf{Thời gian}: Dự kiến kéo dài trong 3–4 tháng.  
\end{itemize}
\section{PHƯƠNG PHÁP NGHIÊN CỨU}
Nghiên cứu này sẽ tập trung vào việc xây dựng và triển khai một hệ thống tích hợp, bao gồm các khía cạnh sau:

\begin{itemize}
    \subsection{Về công nghệ và nền tảng:}
    \begin{itemize}
        \item {Tự động hóa:} Sử dụng các công cụ N8N và Make.com để thiết lập các luồng tự động hóa quy trình.
        \item {Website Thương mại điện tử:} Xây dựng trên nền tảng WordPress kết hợp với các Plugin phù hợp (ví dụ: WooCommerce).
        \item {Chatbot:} Tích hợp và cấu hình Chatbot trên nền tảng Zalo, sử dụng Zalo API để tương tác.
        \item {Nền tảng truyền thông:} Kết nối với các API của các mạng xã hội phổ biến (ví dụ: Facebook Fanpage, Instagram, TikTok) để tự động đăng tải nội dung.
    \end{itemize}

    \subsection{Về chức năng hệ thống:}
    \begin{itemize}
        \item{Tạo và đăng tải nội dung tự động:} Phát triển mô hình tự động tạo video, hình ảnh marketing và bài viết, sau đó tự động đăng lên các kênh mạng xã hội theo lịch trình.
        \item {Quản lý sản phẩm và bán hàng:} Triển khai Website bán điện thoại với các chức năng cơ bản của thương mại điện tử (hiển thị sản phẩm, giỏ hàng, đặt hàng, quản lý đơn hàng đơn giản).
        \item {Chăm sóc khách hàng tự động:} Chatbot Zalo có khả năng giải đáp các câu hỏi thường gặp, tư vấn thông tin sản phẩm (mô tả, giá, tình trạng hàng), và hỗ trợ các yêu cầu cơ bản từ khách hàng.
    \end{itemize}

    \subsection{Về đối tượng và giới hạn:}
    \begin{itemize}
        \item{Đối tượng ứng dụng:} Tập trung vào các doanh nghiệp vừa và nhỏ (SMEs) hoạt động trong lĩnh vực bán lẻ sản phẩm công nghệ, đặc biệt là điện thoại di động.
        \item {Giới hạn:} Nghiên cứu sẽ tập trung vào việc tích hợp các công cụ và nền tảng có sẵn để xây dựng hệ thống tự động hóa. Đề tài không đi sâu vào việc phát triển các thuật toán AI hoặc Engine Chatbot từ đầu mà sẽ tận dụng các giải pháp API và công cụ no-code/low-code hiện có. Hệ thống thanh toán và vận chuyển sẽ được tích hợp ở mức cơ bản hoặc giả lập tùy theo khả năng triển khai trong phạm vi thực tập.
    \end{itemize}
\end{itemize}
\section{BỐ CỤC BÀI VIẾT}
Đồ án thực tập bao gồm 4 chương:
\begin{itemize}
    \item Chương 1: GIỚI THIỆU TỔNG QUAN.
    \item Chương 2: TỔNG QUAN VỀ CƠ SỞ LÝ THUYẾT
    \item Chương 3: KẾT QUẢ - THẢO LUẬN.
    \item Chương 4: KẾT LUẬN VÀ ĐỀ XUẤT (KIẾN NGHỊ).
\end{itemize}
\end{abstract}