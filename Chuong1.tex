\chapter{GIỚI THIỆU TỔNG QUAN}
\section{THÔNG TIN VỀ ĐƠN VỊ THỰC TẬP}
\subsection{ Sơ lược về sự hình thành và phát triển của đơn vị}
Công ty Cổ phần Bảo vệ Thực vật Delta (Delta Corp.) bắt đầu tham gia vào lĩnh vực nông nghiệp từ năm 2009 và không ngừng nỗ lực để khẳng định vị thế của mình trong ngành nông nghiệp Việt Nam. Với tầm nhìn dài hạn và chiến lược phát triển bền vững, công ty chuyên phân phối các mặt hàng thuốc bảo vệ thực vật và sản xuất phân bón trung vi lượng phục vụ trực tiếp cho bà con nông dân. \\
\indent Qua nhiều năm hoạt động, công ty giữ vững phương châm \textit{``Chất lượng sản phẩm là uy tín – Sự hài lòng của khách hàng là thành công''}, đồng thời liên tục mở rộng quy mô sản phẩm từ thuốc đặc trị đến các dòng phân bón dưỡng cây. Các sản phẩm đã phủ sóng tại Đồng bằng sông Cửu Long, miền Đông Nam Bộ và Tây Nguyên, góp phần cải thiện năng suất mùa vụ. \\
\indent Hiện tại, Delta sở hữu gần 50 cán bộ – nhân viên, với môi trường làm việc chuyên nghiệp, đoàn kết, cùng nhau xây dựng mục tiêu phát triển bền vững theo tinh thần:
\begin{itemize}
    \item ``Đóng góp nhỏ – Gặt hái to''.
    \item ``Phát triển nội lực – Vững bước tương lai''.
\end{itemize}

\subsection{ Tổ chức và các lĩnh vực hoạt động của đơn vị}
Hiện nay, các sản phẩm của Delta được phân phối rộng rãi tại nhiều tỉnh thành, đặc biệt là Đồng bằng Sông Cửu Long, miền Đông Nam Bộ và Tây Nguyên. Trong 5 năm tới, Delta định hướng mở rộng thị trường ra miền Trung và miền Bắc, đồng thời chú trọng nghiên cứu các dòng chế phẩm sinh học hướng tới tiêu chí ``Môi trường xanh – sản phẩm sạch''. \\
\indent Bên cạnh thị trường và sản phẩm, công ty cũng chú trọng đào tạo nhân sự, xây dựng đội ngũ vững mạnh cả về chuyên môn lẫn đạo đức nghề nghiệp.

\subsection{ Tổ chức quản lý và sử dụng các nguồn lực của đơn vị}
\subsubsection{a. Nguồn nhân lực}
Delta hiện có gần 50 cán bộ – nhân viên, phân bổ tại các phòng ban: Ban Giám đốc, Phòng Kinh doanh, Phòng Kỹ thuật, Phòng Công Nghệ Thông Tin, Phòng Nghiên cứu \& Phát triển, Phòng Hành chính – Nhân sự, Phòng Kế toán – Tài chính và Đội ngũ thị trường. Công ty chú trọng đào tạo, bồi dưỡng kỹ năng và giữ chân nhân tài.
\subsubsection*{b. Nguồn lực tài chính}
Nguồn vốn ổn định, đảm bảo sản xuất – kinh doanh, phân bổ hợp lý cho đầu tư công nghệ, nghiên cứu sản phẩm, Marketing và đãi ngộ nhân sự.
\subsubsection*{b. Nguồn lực vật chất – kỹ thuật}
Sở hữu nhà máy sản xuất phân bón hiện đại, hệ thống kho bãi đạt chuẩn và phương tiện vận chuyển chuyên dụng.
\subsubsection*{c. Nguồn lực công nghệ – thông tin}
Ứng dụng phần mềm quản lý kho, bán hàng, kế toán, CRM và nền tảng số ( Zalo OA, Website, Fanpage) để tăng hiệu quả vận hành.
\subsubsection*{d. Nguồn lực quan hệ – thị trường}
Mạng lưới hơn 300 đại lý và nhà phân phối, hợp tác với chuyên gia nông nghiệp và tổ chức bảo vệ thực vật để cải tiến sản phẩm.

\subsection{ Cơ cấu tổ chức}
Bộ máy quản lý gồm Ban Giám đốc, Phòng Kinh doanh, Phòng CNTT, Phòng Kỹ thuật – Tư vấn, Phòng R\&D, Hành chính – Nhân sự, Kế toán – Tài chính, và Xưởng sản xuất – kho vận. Các phòng ban phối hợp chặt chẽ, đảm bảo hoạt động trơn tru từ nghiên cứu, sản xuất, phân phối đến chăm sóc khách hàng.

\subsection{ Mô tả sơ đồ hoạt động}
Quy trình hoạt động của Delta bao gồm:
\begin{itemize}
    \item Nghiên cứu sản phẩm mới.
    \item Sản xuất – kiểm định chất lượng.
    \item Phân phối – tiếp cận thị trường.
    \item Chăm sóc khách hàng – hậu mãi.
    \item Phản hồi – cải tiến sản phẩm và dịch vụ.
\end{itemize}

\subsection{ Năng lực sản xuất – phát triển sản phẩm}
Delta có công suất sản xuất 1.000–1.500 tấn/tháng, với hơn 300 dòng phân bón và 100 dòng thuốc bảo vệ thực vật. Công ty liên tục cải tiến công thức, hợp tác với chuyên gia, đồng thời tổ chức hội thảo đầu bờ giúp bà con nông dân tiếp cận và sử dụng hiệu quả sản phẩm. Delta hướng tới mở rộng thị trường trong nước và quốc tế, phát triển bền vững và thân thiện môi trường.
\section{THÔNG TIN VỀ CÔNG VIỆC THỰC TẬP}
Trong quá trình thực tập, sinh viên được giao các công việc liên quan đến lĩnh vực công nghệ thông tin và truyền thông số, cụ thể như sau:

\begin{itemize}
    \item \textbf{Tìm hiểu và nghiên cứu nhu cầu truyền thông số của công ty:} 
    Phân tích hiện trạng hoạt động kinh doanh, nhận thấy công ty chủ yếu bán hàng qua hệ thống đại lý, các kênh trực tuyến chưa được khai thác hiệu quả.
    
    \item \textbf{Đề xuất xây dựng hệ thống kênh truyền thông đa nền tảng:} 
    Bao gồm Zalo Official Account, Facebook Fanpage và Website WordPress để tăng khả năng tiếp cận khách hàng.
    
    \item \textbf{Thiết kế và triển khai Chatbot Zalo:}
    \begin{itemize}
        \item Tự động trả lời câu hỏi về sản phẩm (phân bón, thuốc bảo vệ thực vật).
        \item Gửi tin nhắn quảng bá theo từng chiến dịch Marketing.
        \item Hỗ trợ khách hàng đặt hàng nhanh và chốt đơn tự động.
        \item Kết nối Chatbot với cơ sở dữ liệu sản phẩm của công ty.
    \end{itemize}
    
    \item \textbf{Phát triển nội dung truyền thông:}
    \begin{itemize}
        \item Biên tập và xử lý Video quảng cáo (giới thiệu sản phẩm, hướng dẫn sử dụng) để đăng tải trên Facebook Fanpage.
        \item Thiết kế hình ảnh truyền thông (Poster, Banner, Infographic) bằng Canva/Photoshop và đăng tải trên Website WordPress.
        \item Tối ưu nội dung chuẩn SEO để tăng hiển thị trên Google.
    \end{itemize}
    
    \item \textbf{Xây dựng hệ thống thương mại điện tử tích hợp:}
    \begin{itemize}
        \item Kết nối Chatbot Zalo với hệ thống quản lý đơn hàng.
        \item Tự động hóa việc gửi thông tin đơn hàng từ Zalo về Email hoặc phần mềm quản lý bán hàng.
        \item Hỗ trợ xây dựng giao diện bán hàng trên WordPress (giới thiệu sản phẩm, Form đặt hàng trực tuyến).
    \end{itemize}
    
    \item \textbf{Báo cáo – đánh giá hiệu quả:}
    \begin{itemize}
        \item Thống kê số lượng khách hàng tương tác với Chatbot.
        \item Đánh giá hiệu quả Video quảng cáo trên Facebook (Reach, lượt xem, tương tác).
        \item Đo lường mức độ quan tâm của khách hàng từ Website WordPress.
    \end{itemize}
\section{NỘI QUY AN TOÀN - VỆ SINH LAO ĐỘNG}
\begin{itemize}
    \subsection{Quy định chung:}
    \begin{itemize}
        \item Giữ gìn trật tự, vệ sinh trong phòng làm việc; không ăn uống gần khu vực máy tính, thiết bị mạng.
        \item Sử dụng thiết bị đúng mục đích, không tự ý di chuyển, tháo lắp máy tính hoặc linh kiện khi chưa có sự cho phép.
        \item Luôn tuân thủ giờ giấc làm việc, tôn trọng không gian làm việc chung.
    \end{itemize}

    \subsection{An toàn trong sử dụng thiết bị CNTT:}
    \begin{itemize}
        \item Kiểm tra dây nguồn, thiết bị ngoại vi trước khi sử dụng; tắt máy tính đúng quy trình khi kết thúc công việc.
        \item Không cài đặt phần mềm lạ, không rõ nguồn gốc lên hệ thống của công ty.
        \item Đảm bảo an toàn thông tin: không chia sẻ tài khoản, mật khẩu hoặc dữ liệu nội bộ ra bên ngoài.
    \end{itemize}

    \subsection{Quy định khi bảo trì và lắp ráp thiết bị:}
    \begin{itemize}
        \item Ngắt nguồn điện trước khi tháo lắp, thay thế linh kiện máy tính hoặc thiết bị mạng.
        \item Sử dụng đúng dụng cụ được cấp khi bảo trì, không sử dụng thiết bị cá nhân không đạt chuẩn.
        \item Chỉ thực hiện các công việc kỹ thuật dưới sự hướng dẫn của cán bộ phụ trách.
    \end{itemize}

    \subsection{An toàn tại nơi làm việc:}
    \begin{itemize}
        \item Không để vật cản chắn lối đi, đảm bảo lối thoát hiểm luôn thông thoáng.
        \item Thực hiện phân loại rác thải (rác thải sinh hoạt, rác thải điện tử) đúng quy định.
        \item Giữ thái độ làm việc chuyên nghiệp, hỗ trợ đồng nghiệp khi cần thiết.
    \end{itemize}
\end{itemize}
\end{itemize}
\section{THÔNG TIN LIÊN HỆ}
\begin{itemize}
    \item \textbf{Địa chỉ:} 132K, đường Nguyễn Văn Cừ, Phường An Khánh, Quận Ninh Kiều, Thành Phố Cần Thơ.
    \item \textbf{Địa chỉ:} 0827 26 27 28
\end{itemize}

